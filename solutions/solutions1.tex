\documentclass{amsart}[12pt]


\usepackage{mathpazo}
\linespread{1.2}
\title{MAS-439: Solutions to Problem set 1}



\newcommand{\Q}{\mathbb{Q}}
\newcommand{\Z}{\mathbb{Z}}
\newcommand{\R}{\mathbb{R}}
\newcommand{\C}{\mathbb{C}}

\begin{document}
\maketitle

\section*{Question 1}

Each part gives a subset of a commutative ring.  Determine, with proof, whether or not each subset is a subring.

\subsection*{Part 1} The subset of $\Q$ consisting of numbers with odd denominators, when written in lowest terms.

\begin{proof}
Recall that a subset $S$ is a subring of $R$ if and only if it contains $1_R$, and is closed under addition, taking negatives, and multiplication.

Let $\mathbf{Odd}\subset\Q$ be the subset of $\Q$ consisting of those numbers which have an odd denominator when written in lowest terms; we show that $\mathbf{Odd}$ satisfies these properties, and hence is a subring.


\begin{itemize}
\item \emph{Identity} When written in lowest terms, the identity in $\Q$ is $1/1$ and hence in $\mathbf{Odd}$.  
\item \emph{Negatives} If $a/b\in\mathbf{Odd}$, its negative is $(-a)/b$, which has the same denominator and hence in $\mathbf{Odd}$.
\item \emph{Addition} This is the most delicate; suppose $a/b$ and $c/d$ are both in $\mathbf{Odd}$.  Their sum is $a/b+c/d=(ad+bc)/bd$, which has an odd denominator,  but this expression may not be in lowest terms.  However, when we simplify to lowest terms, the resulting denominator will be a factor of the starting denominator, and hence will still be odd.
\item \emph{Multiplication} If $a/b$ and $c/d$ are in $\mathbf{Odd}$, their produc is $ac/bd$, which has an odd denominator.  Again, this may not be in lowest terms, but when we simplify we will only divide by things.
\end{itemize}
\end{proof}

\subsection*{Part 2} Let $\omega=e^{2\pi i/3}=-1/2+i\sqrt{3}/2$, and let 
$$E=\left\{a+b\omega | a,b\in \Z\right\}\subset\C$$

\begin{proof}
Again, we have to check four things:
\begin{itemize}
\item \emph{Identity} The multiplicative identity $1=1+0\cdot\omega\in E$.
\item \emph{Closed under negatives} If $z=a+b\omega\in E$, then $a,b\in\Z$, and so $-z=-a+(-b)\omega\in E$.
\item \emph{Closed under addition} If $z=a+b\omega$ and $w=c+d\omega$ are both in $E$, then $a,b,c,d\in \Z$, and $z+w=(a+c)+(b+d)\omega\in E$.
\item \emph{Closed under multiplication} If $z=a+b\omega, w=c+d\omega$ are in $E$, then 
$$z\cdot w=ab+(ad+bc)\omega+bd\omega^2$$
 does not immediately appear to be in $E$.

However, since $\omega^3=1$ and $\omega\neq 1$, we see that $0=(\omega^3-1)/(\omega-1)=\omega^2+\omega+1$.  Thus, we may substitute $\omega^2=-1-\omega$ into our expression for $z\cdot w$ to obtain

$$z\cdot w=(ab-bd)+(ad+bc-bd)\omega$$
and since $a,b,c,d\in\Z$, we have $z\cdot w\in E$.
\end{itemize}
\end{proof}


\subsection*{Part 3} Polynomials $f\in\R[x]$ satisfying $f(0)=0$.
\begin{proof}
This is not a subring of $\R[x]$, as it does not contain the identity function 1; 1 evaluated at $x=0$ is just 1.
\end{proof}

\subsection*{Part 4} Polynomials $f\in R[x]$ satisfying $f^\prime(0)=0$.
\begin{proof}
Let $$\mathbf{D}_0\subset\R[x]=\left\{f\in\R[x] | f(0)=0\right\}$$
we show that $\mathbf{D}_0$ is a subring by checking the four properties:
\begin{itemize}
\item \emph{idenity} The identity in the ring $\R[x]$ is the constant polynomial 1.  The derivative of any constant function is 0, and so in particular the derivative of 1 gives 0 when evaluated at 0, and hence is in $\mathbf{D}_0$.
\item \emph{closed under negatives} If $f(0)=0$, then $(-f)(0)=-f(0)=0$.
\item \emph{closed under addition} The derivative is linear; if $f,g$ are both in $\mathbf{D}_0$, then 
$$(f+g)\prime(0)=f^\prime(0)+g^\prime(0)=0+0=0$$
and so $f+g\in\mathbf{D}_0$.
\item \emph{closed under multiplication} This uses the product rule for the derivative: $(fg)^\prime=f^\prime g+fg^prime$.  In particular, using this, if $f,g\in\mathbf{D}_0$, we calculate:
$$(fg)^\prime(0)=f^\prime(0)g(0)+f(0)g^\prime(0)=0\cdot g(0)+f(0)\cdot g(0)=0$$
since anything times 0 is 0.
\end{itemize}
\end{proof}

\section*{Question 2}
Let $R$ be a nontrivial commutative ring.  An element $r\in R$ is a \emph{zero divisor} if there exists $s\in R, s\neq 0$, with $r\cdot s=0$.  Show that for a nontrivial commutative ring $R$, the element $x\in R[x]$ is not a zero divisor.

\begin{proof}
We need to show that for any nonzero element $r=a_0+a_1x+a_2x+\cdots+a_nx^n\in R[x], a_i\in R$, then $x\cdot r$ is not a zero divisor.

If $R$ is a nontrivial ring, then elements of $R[x]$ are zero if and only if each of the $a_i=0$.  Since $r\neq 0$, there is some $i$ with $a_i\neq 0$.  Then we have 
$$x\cdot r=x\cdot a_0+a_1x+\cdots +a_nx^n=a_0x+a_1x^2+\cdots +a_nx^{n+1}$$
and, since $a_i\neq 0$, we have $x\cdot r\neq 0$, and $r$ is not a zero divisor.

This question probably should have included a part $b$; a monic polynomials is one whose leading coefficient is $1$, i.e., $f=x^n+a_{n-1}x^{n-1}+\cdots+a_1x+a_0$.  Show that if $R$ is a nontrivial ring, then any monic polynomial is not a zero divisor in $R[x]$.


\end{proof}



\section*{Problem 3}
A not-necessarily commutative ring $R$ is called \emph{Boolean} if $x^2=x$ for all $x\in R$.

\subsection*{Part 1} Show that in a Boolean ring, $2x=0$.
\begin{proof}
There are a few ways to do this, all depend on expanding out the square of a sum.  Let us consider $(x+1)^2$.  On the one hand, since $R$ is boolean, we have $(x+1)^2=x+1$.  On the other hand, using the distributive rule we have

$$(x+1)^2=x\cdot x+x\cdot 1+1\cdot x+1\cdot 1=x^2+2x+1=3x+1$$
where in the last step we have used $x^2=x$ since $R$ is boolean.

So, $x+1=3x+1$, and hence $2x=0$ for any $x\in R$.

\end{proof}
\subsection*{Part 2} show that all Boolean rings are in fact commutative.

\begin{proof}
We must show that for any $a,b\in R$, we have $ab=ba$.  

Consider $(a+b)^2$.  On the one hand, since $R$ is boolean, this is just $a+b$.  On the other hand, expanding by distributive rule gives

$$(a+b)\cdot(a+b)=a^2+a\cdot b+b\cdot a+b^2=a+b+a\cdot b+b\cdot a$$
where at the end we again used that $R$ is Boolean.

Setting these two expressions for $(a+b)^2$ equal to each other, we that
$a+b=a+b+a\cdot b+b\cdot a$, and subtracting $a+b$ from each side gives the desired result.

\end{proof}
\end{document}
